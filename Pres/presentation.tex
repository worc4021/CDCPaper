\documentclass{beamer}
\usepackage[latin1]{inputenc}
\usepackage[english]{babel}
\usepackage{hyperref, amsthm}
\usepackage{amsmath,amssymb}
\usepackage{mathrsfs}
\usepackage{xcolor,ifthen,animate,verbatim}
\usepackage{lpic}

\usepackage{tikz}
\usetikzlibrary{arrows,positioning,patterns}

\tikzset{
    at xy split/.style 2 args={
        at={(#1,#2)}
    },
    a/.style={circle, draw=red},`'
    b/.style={rectangle, draw=blue}
}

\graphicspath{{../Pix/}}

\usecolortheme{sidebartab}
\usefonttheme{serif}


\mode<presentation>
\title{Robust positively invariant sets for state dependent and scaled disturbances}
\subtitle{54th IEEE Conference on Decision and Control, Osaka}
\author{Rainer M. Schaich \and Mark Cannon}
\date{18.~December 2015} 


\theoremstyle{plain}
\newtheorem{thm}{Theorem}



\definecolor{mygray}{cmyk}{75,64,0,0}
\definecolor{darkblue}{rgb}{0,200,117}

\setbeamertemplate{blocks}[rounded][shadow=true]
\setbeamercolor{structure}{fg=mygray}
\setbeamercolor{block title}{use=structure, fg=structure.fg, bg=structure.fg!20!bg}
\setbeamercolor{block body}{parent=normal text, use=block title, bg=block title.bg!50!bg}
\setbeamercolor{plots}{fg=black,bg=white}


\setbeamertemplate{footline}
{%
  \leavevmode%
  \hbox{%
  % \begin{beamercolorbox}[wd=.2\paperwidth,ht=2.25ex,dp=1ex,left]{author in head/foot}%
  %   \usebeamerfont{author in head/foot}\hskip3pt\insertdate
  % \end{beamercolorbox}%
  % \begin{beamercolorbox}[wd=.6\paperwidth,ht=2.25ex,dp=1ex,center]{title in head/foot}%
  %   \usebeamerfont{title in head/foot}\inserttitle
  % \end{beamercolorbox}%
  \begin{beamercolorbox}[wd=\paperwidth,ht=2.25ex,dp=1ex,right]{date in head/foot}%
	\hspace*{2em}
    \insertframenumber{} / \inserttotalframenumber\hspace*{2ex} 
  \end{beamercolorbox}}%
  \vskip0pt%
}

\setbeamertemplate{navigation symbols}{}

\AtBeginSection[]
{
	\begin{frame}
		\frametitle{Outline}
		\tableofcontents[currentsection]
		% \nopage
	\end{frame}
}


\begin{document}
	{
	\usebackgroundtemplate{%
	  \vbox to \paperheight{\vfil\vspace{5cm}\hbox to .5\paperwidth{\hfil%
	  \includegraphics[scale=.7]{../../../ox_logo_cmyk_blue_pos2}%
	  \hspace{1cm}\hfil}\vfil}%
	  \vbox to \paperheight{\vfill\vspace{5cm}\hbox to \paperwidth{\hspace{.1\paperwidth}
	  %
	  \hfill}\vfill}
	}
		\begin{frame}
			\titlepage
		\end{frame}
	}


\begin{frame}
\frametitle{Motivation}
\end{frame}

\begin{frame}
\frametitle{Outline}
\tableofcontents
\end{frame}

\section{The Maximal Robust Positive Invariant Set}
\begin{frame}
\frametitle{The Maximal Robust Positive Invariant Set}
For a constrained linear system
\[
x^+ = \Phi x + v,
\]
with $x,x^+\in\mathcal X_0$ and $v\in\mathscr V$\only<2>{\textcolor{red}{$(x)$}} the \emph{maximal robust positive invariant} 
(MRPI) \emph{set} is defined as the largest set for which
\only<1>{
\[
\mathscr X^\infty = \left\{x\in\mathcal X_0 : \Phi x + v \in \mathscr X^\infty\;\forall v\in\mathscr V\right\}
\]}
\only<2>{
\[
\mathscr X^\infty = \left\{x\in\mathcal X_0 : \Phi x + v \in \mathscr X^\infty\;\forall v\in\mathscr V{\color{red}{(x)}}\right\}
\]}
holds.
\end{frame}

\section{Parametrically Convex Sets}
\begin{frame}
\frametitle{Parametrically Convex Sets}
\begin{block}{Definition: Parametric Convexity}
A continuous set-valued map $T: X\rightarrow\mathcal P(Y)$, $X\ni x\mapsto T(x)\subseteq Y$ is called \emph{parametrically convex}, if for all $x_1,x_2\in X$and all $0\leq\lambda\leq1$ $T(\lambda x_1 + (1-\lambda)x_2)\subseteq \lambda T(x_1) \oplus (1-\lambda)T(x_2)$ holds.
\end{block}


\uncover<2>{
\begin{figure}
\centering
\begin{lpic}{../Pix/pConvexSet}
\lbl[tl]{54,12;$x$}
\lbl[bl]{25,25;$T(x)$}
\end{lpic}
\caption{Scalar parametrically convex set}
\end{figure}}
\end{frame}
 
\begin{frame}
\frametitle{Parametric Pontryagin Difference}
\begin{block}{Definition: Pontryagin Difference}
For a set $S\subseteq X$ and a parametrically convex set-valued map $T:X\rightarrow \mathcal P(X)$
the \emph{parametric Pontryagin difference} $S\ominus T(S)$ is given by the 
$$
S\ominus T(S) = \bigl\{x\in X : \{x\}\oplus T(x)\subseteq S \bigr\}
$$
\end{block}

\begin{block}<2->{Lemma}
The parametric Pontryagin difference $S\ominus T(S)$ is convex \emph{iff} $T$ is parametrically convex.
\end{block}
\end{frame}

\begin{frame}
\frametitle{Parametric Pontryagin Difference}

\begin{center}
\begin{tikzpicture}[scale=1.5]
	\usebeamercolor{plots}
	\draw[ultra thick, fg] (-2,2) -- (2,2) -- (2,-2) -- (-2,-2) -- cycle;
	\only<1-2>{
	\draw[-latex'] (-3,.5) to[out=0,in=180] (-2,0);
	\draw (-3,.5) node[left] {{$S$}};}
	\only<3->{
	\draw[bg,-latex'] (-3,.5) to[out=0,in=180] (-2,0);
	\draw (-3,.5) node[left] {{$\hphantom{S}$}};}
	\only<2->{
	\draw[fg] plot[domain=-2:2] (\x, {1.8 -.1*pow(\x,2))});}
	\only<2>{
	\draw[-latex'] (0,0) to[out=75,in=-75] (0,1.8);
	\draw (0,0) node[below,rotate=0] {$\biggl\{x: \begin{array}{l}\{x\}\oplus T(x)\subseteq S \wedge \\ \not\exists \epsilon>0\; \{x\}\oplus \epsilon T(x) \subseteq S \end{array} \biggr\}$};
	\draw (.6,1.656) -- (1.2,2) -- (1.8,1.656) -- (1.2,1.312) -- cycle;
	\fill[pattern = north east lines, opacity=.5] (.6,1.656) -- (1.2,2) -- (1.8,1.656) -- (1.2,1.312) -- cycle;
	\draw (-2,2) -- (-1.4,1.4) -- (-2,.8) -- (-2.6,1.4) -- cycle;
	\fill[pattern = north east lines, opacity=.5] (-2,2) -- (-1.4,1.4) -- (-2,.8) -- (-2.6,1.4) -- cycle;
	\draw[-latex'] (1.5,.7) to[out=120,in=-70] (1.3,1.5);
	\draw (1.5,.7) node[below] {$T(x)$};
	}
	\only<3->{
	\draw[fg] plot[domain=-1.79999:-1.4] (\x, {sqrt(18+10*\x)});
	\draw[fg] plot[domain=-1.79999:-1.4] (\x, {-sqrt(18+10*\x)});
	\draw[fg] plot[domain=-2:2] (\x, {-1.8 + .1*pow(\x,2)});
	\draw[fg] plot[domain=1.4:1.8] (\x, {sqrt(18-10*\x)});
	\draw[fg] plot[domain=1.4:1.8] (\x, {-sqrt(18-10*\x)});
	}
	\only<4>{
	\fill[pattern = north east lines, opacity=0.8] (-1.55744,1.55744) -- plot[domain=-1.55744:1.55744] (\x,{1.8 -.1*pow(\x,2))}) -- (1.55744,1.55744) -- plot[domain=1.55744:1.79999] (\x, {sqrt(18-10*\x)}) -- (1.8,0) -- plot[domain=1.79999:1.55744] (\x, {-sqrt(18-10*\x)}) -- (1.55744,-1.55744) -- plot[domain=1.55744:-1.55744] (\x, {-1.8 + .1*pow(\x,2)}) -- (-1.55744,-1.55744) -- plot[domain=-1.55744:-1.79999] (\x, {-sqrt(18+10*\x)}) -- (-1.8,0) --plot[domain=-1.79999:-1.55744] (\x, {sqrt(18+10*\x)}) -- (-1.55744,1.55744) -- cycle;
	\draw[-latex'] (-2.1,.5) to[out=0,in=160] (-1.5,0);
	\draw (-2.1,.5) node[left] {{$S\ominus T(S)$}};
	}
\end{tikzpicture}
\end{center}

\end{frame}

\begin{frame}
\frametitle{Piecewise affine setsk}


\end{frame}

\end{document}
