\documentclass[letterpaper, 10pt, conference]{ieeeconf} % Change font size to 9pt

\IEEEoverridecommandlockouts
\overrideIEEEmargins

\usepackage{graphics}
\usepackage{epsfig}
% \usepackage{mathptmx}
% \usepackage{times}
\usepackage{amsmath}
\usepackage{amssymb}
\usepackage{cite}
\usepackage{lpic}
\usepackage{overpic}
% \usepackage{blkarray}
\usepackage{mathrsfs}
\usepackage{todonotes}

\newtheorem{thm}{Lemma}[section]
\newtheorem{rem}[thm]{Remark}
\newtheorem{defn}[thm]{Definition}
\newtheorem{assn}{Assumption}
\newtheorem{algo}[thm]{Algorithm}

\providecommand{\norm}[1]{\left\|#1\right\|}
\providecommand{\abs}[1]{\left|#1\right|}
\providecommand{\conv}{\text{conv}}

\usepackage{xcolor}
\def\edit{\textcolor{blue}}
\allowdisplaybreaks[4]


\DeclareFontFamily{U}{mathx}{\hyphenchar\font45}
\DeclareFontShape{U}{mathx}{m}{n}{
      <5> <6> <7> <8> <9> <10> gen * mathx
      <10.95> mathx10 <12> <14.4> <17.28> <20.74> <24.88> mathx12
      }{}
\DeclareSymbolFont{mathx}{U}{mathx}{m}{n}
\DeclareFontSubstitution{U}{mathx}{m}{n}
\DeclareMathSymbol{\temp}{\mathbin}{mathx}{'341}
\newcommand{\bigominus}{\raisebox{10pt}{$\temp$}}

\graphicspath{{./Pix/}}

\begin{document}
%  \title{Advances in Robust Positively Invariant Set Computation}
  \title{Robust positively invariant sets for state dependent and scaled disturbances}

\author{Rainer M. Schaich\textsuperscript{\dag} %
         and Mark Cannon\textsuperscript{\dag,\ddag}%
\thanks{\textsuperscript{\dag} Department of Engineering Science, University of Oxford, OX1 3PJ.}%
\thanks{\textsuperscript{\ddag} Corresponding author, 
        \texttt{mark.cannon@eng.ox.ac.uk}.}
}
\newcommand{\note}[1]{\todo[inline]{#1}}

\maketitle

\begin{abstract} 
  This paper introduces methods of deriving and computing maximal robust positively invariant sets 
  for linear discrete time systems with additive model uncertainty. Two types of uncertainty are 
  considered: state dependent uncertainty, which can handle multiplicative parametric model uncertainty 
  as well as linearisation errors for nonlinear systems, and scaled
  sets of uncertainty. We provide 
  a framework for analysing both types of uncertainty with illustrative examples.
\end{abstract}

\begin{keywords}
constrained control, positively invariant sets, robust control
\vskip-\baselineskip
\end{keywords}

\section{Introduction}
The analytic properties of disturbance invariant sets have been widely
studied (e.g.~\cite{blanchini:2007}) 
since their introduction in~\cite{Glover:1971,bertsekas71}. A disturbance invariant set $\mathscr X$ is a subset of the 
state constraint set $\mathcal X_0\subseteq\mathbb R^n$ containing
states $x$ of the perturbed linear system
\begin{equation}\label{eq:system:equation}
	x^+ = \Psi x + v
\end{equation}
(where $\Psi$ is Hurwitz, $x,x^+\in\mathcal X_0$, and 
$v\in\mathscr V$), such that the successor state $x^+$ is contained in $\mathscr X$
%the disturbance invariant set 
for all possible realisations of the 
unknown disturbance $v$. Thus a disturbance invariant set is a type of robust positively invariant (RPI) set. 
This is summarised in the implicit definition
\begin{equation}\label{eq:definition:MRPI:set:state:dependent}
	\mathscr X = \{x:\Psi x + v\in\mathscr X\; \forall v\in\mathscr V\}.
\end{equation}

In many applications we are interested in the largest such set, denoted $\mathcal X^\infty$, which is given 
by the union of all RPI sets $\mathscr X\subseteq\mathcal X^\infty$. This set is known as the maximal robust 
positively invariant (MRPI) set and has been used extensively, for example, to define terminal regions in 
robust model predictive control~\cite{mayne:2000}. Although analytical properties
of these sets were derived at the time of their introduction, algorithms for numerically computing MRPI sets 
were introduced much later~\cite{Blanchini:1994,DeSantis:1994,Kolmanovsky:1998}. 
Earlier algorithms did not guarantee finite determinability of $\mathcal X^\infty$, see e.g.~\cite{Blanchini:1990}, 
or were not guaranteed to produce the maximal RPI set, e.g.~\cite{Blanchini:1991}. 
%
However for the case of disturbances belonging to a fixed set $\mathscr V$, methods of computing 
$\mathcal X^\infty$ are now well established~\cite{blanchini:2007}.

This paper considers sets of disturbances that depend on
parameters. First we consider state-dependent sets,
\begin{equation}\label{eq:PWA:disturbance:set}
	\mathcal V(x) = \left\{v: Gv\leq H(x)\right\},
\end{equation}
where $G$ is a real matrix,
$H(x)$ is a convex piecewise affine function, and the inequality applies elementwise. For the case that $\mathscr V=\mathcal V(x)$ we
provide conditions for convexity and finite determination of $\mathcal X^\infty$ and discuss 
computation of $\mathcal X^\infty$. The case of disturbances depending on states (or control inputs) 
is considered in~\cite{Kuntsevich:1995,rakovic06} but the convexity
analysis of this paper and the associated computational approach are novel. 
State dependent disturbance constraints can account for linearisation errors, as shown in section~\ref{sec:first:example}, 
as well as more general multiplicative parametric model uncertainty.

The second case considered is that of scaled disturbance sets of the form
\begin{equation}
  \mathcal V(\theta) = \{v:Gv\leq(1+\theta){\bf{1}}\},
\end{equation}
for scalar $\theta>-1$. For $\mathscr V = \mathcal V(\theta)$ we compute sets $\mathcal Z^\infty$ in 
$(x\times\theta)$-space with the property that 
the intersection, $\mathcal Z^\infty\vert_{\hat\theta}$, with the subspace on which $\theta=\hat\theta$ 
for given $\hat\theta$ is an MRPI set. To our best knowledge, this setup has not been 
considered in the literature. This type of disturbance constraint can be used to study the sensitivity of
the MRPI set to changes in the disturbance \emph{strength}.

The structure of the paper is as follows: We introduce the notion of \emph{parametric convexity} In
Section~\ref{sec:parametrically:convex:set:operations} and discuss its implications, in particular as a condition 
for convexity of the MRPI set. The algorithm to compute the MRPI set for state dependent disturbance
constraints and its finite determinability are discussed in Section~\ref{sec:state:dep:MRPI}.
Section~\ref{sec:first:example} illustrates the algorithm using a nonlinear model of a magnetically 
levitated ball. The case of scaled disturbance constraints is discussed in Section~\ref{sec:MRPI:parametrised},
which gives the algorithm for computing the parametrised MRPI set and
proves its finite determinability. Section~\ref{sec:second:example} illustrates the use of
parametrised MRPI sets using the levitating ball example, and illustrates a robustness analysis of the parametrised
MRPI sets with a numerical example. Section~\ref{sec:conclusions}
provides conclusions.

Throughout this paper we refer to sets that can be represented as intersections of finite numbers of half 
spaces as polyhedra, and to bounded polyhedra as polytopic sets. For sets $\mathcal A, \mathcal B 
\subset\mathbb R^n$, the Minkowski set addition is denoted $\mathcal A\oplus\mathcal B = \{c : c = a + 
b\; \forall\,a\in\mathcal A,\, b\in\mathcal B\}$. We use ${\bf{1}}$ to denote the column vector of ones 
in appropriate dimensions, and $\wedge$ is the logical AND operator, index sets are denoted by $\mathcal I
\subset\mathbb N$. 
%
%
%
\section{Parametrically convex set operations}\label{sec:parametrically:convex:set:operations}
This section discusses sets that depend on a parameter such as the state of \eqref{eq:system:equation} 
(so called point-to-set maps, see~\cite{Hogan:1973}), and extends the existing set algebra~\cite{blanchini:2007} to 
accommodate such sets. We present the general case first, then illustrate it with the example 
of a pointwise polyhedral set in the form of (\ref{eq:PWA:disturbance:set})
%
%
    \begin{defn}[Parametric Convexity]\label{def:parametric:convexity}
      Let $X\subseteq\mathbb R^n, Y\subseteq\mathbb R^m$, let $\mathcal P(Y)$ denote the power set of $Y$, 
      and $T:X\rightarrow \mathcal P(Y)$, $X\ni s\mapsto T(s)\subset Y$ be a 
      continuous point-to-set map. The map $T$ is called \emph{parametrically convex} if it satisfies
      \begin{equation}\label{eq:pconvexdef}
        T(\lambda s_1 + (1-\lambda)s_2)\subseteq\lambda T(s_1) \oplus (1-\lambda) T(s_2)
      \end{equation}
      for all $s_1,s_2\in X$ and $\lambda\in[0,1]$.
    \end{defn}
%
The difference of sets can be defined for parametrised sets analogously to the Pontryagin difference~\cite{Kolmanovsky:1998} as follows:
%
\begin{defn}[Parametric Pontryagin Difference]\label{def:parametric:pontryagin:difference}
Let $S\subseteq X$ and let $T:X\to\mathcal P(X)$ be a continuous point-to-set map,
then the \emph{parametric Pontryagin difference} $S\ominus T(S)$ is 
      \begin{equation}\label{eq:definition:parametric:pontryagin:difference}
        S\ominus T(S) = \left\{x\in X: \{x\} \oplus T(x)\subseteq S\right\},
      \end{equation}
      where $T(S)$ denotes the image of $S$ under the map $T$. 
    \end{defn}
%
Notice that, for a constant map $T$, the definition~\eqref{eq:definition:parametric:pontryagin:difference} is equivalent to the 
well-known Pontryagin difference.
For the parametric Pontryagin difference of a convex set and a parametrically convex map we have the following result.
%
\begin{thm}\label{thm:convexity:of:pontryagin:difference}
  Let $S\subseteq X$ be a convex set and let $T:X\rightarrow\mathcal P(X)$ be a parametrically convex point-to-set
  map, then $S\ominus T(S)$ is convex. 
\end{thm}
%
\begin{proof}
Define $ Z =  S\ominus T( S)$ and let $z_1,z_2\in Z$, then
by definition of the parametric Pontryagin difference, we have
%
\[
        \{z_i\} \oplus T(z_i) \subseteq S,\; i=1,2.
\]
%
To see that $ Z$ is convex we show that line segments between
all possible $z_1$ and $z_2$ are subsets of $ Z$, i.e.~for all $\lambda \in [0,1]$,
\[\begin{aligned}
  \{ \lambda z_1 + (1-&\lambda)z_2
  \}\oplus T\left( \lambda z_1 + (1-\lambda)z_2\right)\\
  \subseteq&\left\{ \lambda z_1 + (1-\lambda)z_2
  \right\}\oplus \lambda T(z_1) \oplus (1-\lambda)
  T(z_2)\\
  \subseteq &\lambda\underbrace{(\{z_1\}\oplus T(z_1))}_{\subseteq S}\oplus
  (1-\lambda)\underbrace{(\{z_2\}\oplus T(z_2))}_{\subseteq S}\\
  \subseteq& Z
\end{aligned}\]
%
where the last inclusion follows from the convexity of $\mathcal S$.
\end{proof}
%
%
Now consider a parametrically convex set of the form of~\eqref{eq:PWA:disturbance:set}.
\begin{thm}\label{thm:convex:parametric:set}
  The point-to-set map $\mathcal V (x)$ defined by~\eqref{eq:PWA:disturbance:set} is parametrically 
  convex for all $x\in \mathcal X$ if $H(x)$ is elementwise convex in $x\in \mathcal X\subset 
  \mathbb R^n$ (i.e.~if each element $H_i(x)$ of $H(x)$ satisfies $H_i(\lambda x_1+(1-\lambda)x_2)\leq 
  \lambda H_i(x_1)+(1-\lambda)H_i(x_2)$ for all    $\lambda\in[0,1]$ and $x_1, x_2\in\mathcal X$).
\end{thm}
%
\begin{proof}
To show that $\mathcal V(\lambda x_1 + (1-\lambda)x_2)\subseteq \lambda\mathcal V(x_1) \oplus(1-\lambda)
\mathcal V(x_2)$ for all $\lambda \in [0,1]$ and $x_1, x_2\in\mathcal X$ we note that
%
\begin{align*}
  \mathcal V&(\lambda x_1 + (1-\lambda)x_2)\\
  =& \{v:\; G v \leq H(\lambda x_1 + (1-\lambda)x_2)\}\\
  \subseteq& \{v:\;Gv\leq\lambda H(x_1)+(1-\lambda) H(x_2)\}\\
  =&\{v:\;Gv\leq\lambda H(x_1)\}\oplus\{v
  :\;Gv\leq(1-\lambda)H(x_2)\}\\
  =&\lambda\mathcal V(x_1)\oplus(1-\lambda)\mathcal V(x_2).
  \end{align*}
\vskip-1.5\baselineskip
\end{proof} 
%
%
\def\genmat{\Xi} \def\genvec{\xi}
Lemmas~\ref{thm:convexity:of:pontryagin:difference} and~\ref{thm:convex:parametric:set} imply that, 
if $\mathcal V(x)$ is defined by \eqref{eq:PWA:disturbance:set} in terms of an elementwise convex 
function $H(x)$, then the parametric Pontryagin difference $\mathcal X\ominus \mathcal V(\mathcal X)$
is a convex set.
As we show in the next section, if $\mathcal X$ is polytopic and $\mathcal 
V(x)$ is pointwise polytopic, then $\mathcal X\ominus\mathcal V(\mathcal X)$ is 
also a polytope. 
%
%
%
%
\section{Maximal Robust Positively Invariant Sets for State Dependent Disturbance}\label{sec:state:dep:MRPI}
In this section we describe an iterative algorithm to compute the MRPI set~\eqref{eq:definition:MRPI:set:state:dependent} 
for a linear system~\eqref{eq:system:equation} subject to disturbance
$v\in\mathcal V(x)$ where $\mathcal V(x)$ is a parametrically convex point-to-set map defined 
by~\eqref{eq:PWA:disturbance:set} with a convex piecewise affine $H(x)$.
Notice that the set $\mathcal V(x)$ is pointwise polytopic set~\eqref{eq:definition:MRPI:set:state:dependent} 
for a linear system~\eqref{eq:system:equation} subject to disturbances
$v\in\mathcal V(x)$ where $\mathcal V(x)$ is a parametrically convex point-to-set map defined 
by~\eqref{eq:PWA:disturbance:set} with a convex piecewise affine function $H(x)$.
We assume (without loss of generality) that the set $\mathcal V(x)$ is pointwise compact and polytopic 
for finite $x\in\mathcal X_0$ and can hence be represented as the 
convex hull of its vertices $\mathcal V(x) = \conv\{v_i(x)\}$. Since ${\mathcal{V}}(x)$
has a piecewise affine dependence on $x$ the vertices $v_i(x)$ are also piecewise affine in $x$.
The set $\mathcal X^\infty$ as defined in~\eqref{eq:definition:MRPI:set:state:dependent} 
is required to satisfy $\Psi x + v\in\mathcal X^\infty$ for all
$x\in\mathcal X^\infty$ and $v\in\mathcal V(x)$. To compute the MRPI set we start from the given 
state constraint set
\[
\mathcal X_0 = \{x:\;\genmat_{0,i}x\leq \genvec_{0,i}\,\forall i\in\mathcal I_0\}
\]
and
recursively \emph{cut off} points that cannot satisfy the invariance condition~\eqref{eq:definition:MRPI:set:state:dependent}
by iteratively introducing constraints that exclude all points for which the successor state can lie outside 
$\mathcal X_0$. The first iteration enforces the constraint:
%
\[
\begin{split}
	&\genmat_{0,i}(\Psi x + v)\overset{!}{\leq}\genvec_{0,i}\;\forall v\in\conv\{v_i(x)\}\\
	&\genmat_{0,i}\Psi x + \max_{v\in\mathcal V(x)} \genmat_{0,i} v \leq \genvec_{0,i}\\
	&\genmat_{0,i}\Psi x + \underbrace{\max_{j} \genmat_{0,i} v_j(x)}_{=v_{0,i}^\ast(x)} \leq \genvec_{0,i}.
\end{split}
\]
%
for each $i\in \mathcal I_0$.
Notice that $v_{0,i}^\ast(x)$ is not necessarily given by a unique maximiser, 
but is the solution of a multi-parametric linear program and hence will be given 
by a vertex $v_i(x)$ of $\mathcal V(x)$ for each $x$ on that facet.
Since each vertex is a piece-wise affine function of $x$, the maximum $v_{0,i}^\ast(x)$ will also
be piece-wise affine
and therefore the set $\mathcal X_1=\mathcal X_0 \cap \{x:\genmat_{0,i}\Psi x + v_{0,i}^\ast(x) \leq 
\genvec_{0,i}\forall i\in\mathcal I_0\}$
has the representation $\mathcal X_1 = \{x:\genmat_{1,i}x\leq\genvec_{1,i}\,\forall i\in\mathcal I_1\}$.
The next iterate is defined by
\begin{align*}
	\mathcal X_2 &= \mathcal X_1 \cap \\ &\{x:\genmat_{0,i}\Psi(\Psi x + v) + v_{0,i}^\ast(x)\leq\genvec_{0,i}\,
	\forall i\in\mathcal I_0,v\in\mathcal V(x)\}\\
  &= \mathcal X_1 \cap \{x: \genmat_{0,i}\Psi^2 x + v_{1,i}^\ast(x) + v_{0,i}^\ast(x)\leq\genvec_{0,i}\,\forall 
	i\in\mathcal I_0\}
\end{align*}
%
and at the $(k+1)$st iteration we have
%
\[
	\mathcal X_{k+1} = \mathcal X_k\cap \{x:\genmat_{0,i}\Psi^k x + \sum_{l=0}^{k-1}v_{l,i}^\ast(x)
	\leq\genvec_{0,i}\,\forall i\in\mathcal I_0\},
\]
%
where 
%
\begin{align*}
	v_{l,i}^\ast(x)&=\max_j \ \genmat_{0,i}\Psi^{l-1}v_j(x)
	 = \begin{array}[t]{rl} \displaystyle\max_v & \genmat_{0,i}\Psi^{l-1}v \\ \text{s.t.}& v\in\mathcal V(x)
   \end{array} \\ 
   &= \begin{array}[t]{rl} \displaystyle\max_{\tilde v} & \genmat_{0,i}\tilde v\\ \text{s.t.}& \tilde v\in 
   \Psi^{l-1}\mathcal V(x) \end{array}
\end{align*}
%
In closed form the iterates can be expressed as
%
\begin{equation}\label{eq:set:iteration:state:dependent:constraints}
\begin{split}
	\mathcal X_{k+1} =& \mathcal X_k\cap\left(\Psi^{-1}\mathcal X_k \ominus \Psi^{k-1}\mathcal V(\mathcal X_k)\right)
	=\mathcal X_k\cap D_k \\
	=& \bigcap_{0\leq l\leq k+1}\left( \Psi^{-l} \mathcal X_0 \underset{0 \leq i\leq l-1}{\bigominus} 
  \Psi^i \mathcal V(\mathcal X_{l-1})\right).
\end{split}\end{equation}
%
Notice that we do not require $\Psi$ to be invertible, in~\eqref{eq:set:iteration:state:dependent:constraints} $\Psi^{-1}\mathcal X_k$
merely denotes the preimage of $\mathcal X_k$ under the linear map $\Psi$.
We will use~\eqref{eq:set:iteration:state:dependent:constraints} to prove the finite determinability of
$\mathcal X^\infty$, namely that there exists a finite number $N$ such that $x\in\mathcal X_N$ implies
$\Psi x + v \in\mathcal X_N$ for all $v\in\mathcal V(x)$ and hence $\mathcal X_N$ is robustly positively invariant.
%
\begin{thm}\label{thm:finite:MRPI:set:state:dependable}
Let the state constraint set $\mathcal X_0$ be contained in a band: $\mathcal X_0\subseteq B=\{x:\Gamma x\leq{\bf{1}}\wedge 
-\Gamma x\leq{\bf{1}}\}$, let the pair $(\Psi,\Gamma)$ be observable and let $\mathcal V(x)$ be defined 
by~\eqref{eq:PWA:disturbance:set} in terms of a piecewise affine function $H(x)$. Then $\mathcal X_N\subseteq \mathcal 
X_{N+1}$ for a finite $N$, and hence the MRPI set $\mathcal X^\infty =\mathcal X_N$ is a polytope.
\end{thm}
%
\begin{proof}
From~\eqref{eq:set:iteration:state:dependent:constraints} it is clear that $\mathcal X^\infty = \emptyset$ 
if $\mathcal X_k =\emptyset$ for any $k\geq 0$. For the remainder of the proof we assume that the MRPI set 
$\mathcal X^\infty$ is non-empty. For this we require that $\bigcup_{s\in\mathcal S}\mathcal V(s)$ is 
compact on any compact set $\mathcal S\subset\mathbb R^n$.

The proof has two main steps: First we prove that $\mathcal X_p$ is compact for $p\leq n$
where $n$ is the dimension of the state $x$. The second step is to prove that in~\eqref{eq:set:iteration:state:dependent:constraints}
the set $D_k$ grows exponentially, i.e. that for any given compact set $\mathcal C$ there exists
finite $N$ such that $\mathcal C\subseteq D_{N}$. The proof is concluded by setting 
$\mathcal C = \mathcal X_{N}$ and deducing that $\mathcal X^\infty = \mathcal X_N$. 

For the first step, notice that observability of $(\Psi,\Gamma)$ is equivalent to the observability matrix 
$\Omega$ having full rank, i.e.\ $\mathrm{rank}(\Omega) = \mathrm{rank}(\begin{bmatrix} \Gamma^T \ \cdots \ 
(\Gamma\Psi^{n-1})^T\end{bmatrix}) = n$.
But  this implies that the set 
%
\[
\mathcal P_{n} = \{x: 
\Omega x\leq{\bf{1}}\wedge-\Omega x\leq{\bf{1}}\} = \bigcap_{0\leq l\leq n-1} \Psi^{-l} B
\]
%
is bounded, and since $\mathcal X_k\subseteq \mathcal P_k$ for all $k$, 
the set $\mathcal X_{n}$ is also bounded. To show that $D_k$ grows exponentially we use the compactness 
of $\mathcal V(\mathcal X_{n})$. Let $K_1$ denote the smallest ball that contains $\mathcal V(\mathcal X_{n})$ and let
$\rho$ denote the spectral radius of $\Psi$. Furthermore, let $K_2$
be the biggest ball that is contained in $\mathcal X_{0}$. From~\eqref{eq:set:iteration:state:dependent:constraints}, 
$D_k$ has the representation
%
\begin{equation}
D_k = \underbrace{\Psi^{-k}\mathcal X_0}_{=\mathcal S_1} \ominus \underbrace{\biggl(\bigoplus_{i\leq k-1} 
\Psi^i\mathcal V(\mathcal X_k)\biggr)}_{=\mathcal S_2}.
\end{equation}
%
Since $\Psi$ is asymptotically stable we have $\rho<1$ and therefore $\rho^{-l}K_1\subseteq\Psi^{-l} K_1 
\subseteq \Psi^{-l}\mathcal X_0 = \mathcal S_1$, implying exponential growth of $D_k$ provided $\mathcal S_2$ 
is bounded. To bound $\mathcal S_2$ we note that for $k\geq n$ we have $\mathcal S_2\subseteq 
\bigoplus_{0\leq i\leq k-1} \rho^i K_2 \subseteq \frac{1}{1-\rho} K_2$. Thus $D_k$ grows exponentially, whereas 
$\mathcal X_k$ is contained in a ball of finite radius, and hence we conclude from~\eqref{eq:set:iteration:state:dependent:constraints} that 
$\mathcal X_{k+1} = \mathcal X _k$ for all $k\geq N$ and therefore $\mathcal X _N = \mathcal X^\infty$ for finite $N$.
\end{proof}
%
We have seen that we can compute a the maximal robust positively invariant set $\mathcal X^\infty$ for linear 
systems with state dependent constraints. In the next section we will illustrate the concept with an 
example.
%
%
%
\section{Example}\label{sec:first:example}
%
%
\begin{figure}
\centering
\begin{overpic}[scale=0.75]{levitatingBall}
\put(30,5){$m g$}
\put(30,41){$c\frac{i^2}{x^2}$}
\put(78,28){$y$}
\put(90,89){$i$}
\end{overpic}
% \begin{lpic}[scale=0.5]{levitatingBall}
% \lbl[tr]{25,3; $m g$}
% \lbl[br]{25,25; $c\frac{i^2}{x^2}$}
% \lbl[bl]{49,17; $y$}
% \lbl[bl]{56,55; $i$}
% \end{lpic}
\vspace{-2mm}
\caption{Levitating ball system.}
\label{fig:levitating:ball}
\end{figure}
%
%
%
In this section we discuss the calculation of the MRPI set for a linearised simplified model of the magnetic 
levitation system depict in figure~\ref{fig:levitating:ball}. The system dynamics for the ball are given
by $m \ddot y = m g - c\frac{i^2}{y^2}$, where $m,g,c,i$ and $y$ denote the mass of the ball, the gravitational
constant, a constant factor, the current and the distance between the coil and the centre of the ball respectively.
For illustration purposes we neglect inductive dynamics and use the current $u=i$ as an input and the position
$y$ and its first derivative $\dot y$ as the states, i.e. $x = (y,\dot y)^T$. We find that any equilibrium has 
$\dot{y}=0$ and $u=\sqrt{\frac{gm}{c}} y$ for any positive position $y>0$. Linearising the nonlinear differential 
equation $\dot x = f(x,u)$ around an equilibrium point $(\hat x, \hat
u)$ gives the approximate linear model 
%
\begin{equation}
	 \Delta\dot{x} = \underbrace{\left(\begin{array}{cc}
	0 & 1 \\ \frac{2c\hat u^2}{m\hat x_1^3} & 0
	\end{array}\right)}_{\frac{\partial f}{\partial x}(\hat x,\hat
      u)}\Delta x 
+ \underbrace{\left(\begin{array}{c}
	0 \\ - \frac{2c\hat u}{m\hat x_1^2}
	\end{array}\right)}_{\frac{\partial f}{\partial u}(\hat x,\hat
      u)}\Delta u
\end{equation}
%
where $\Delta u = u -\hat{u}$ and $\Delta x \approx x-\hat{x}$.
We derive the discrete time dynamics with sampling rate $T_s$ using
the Euler formula $x^+=x+T_s f(x,u) =:\tilde f(x,u)$ giving
\[
\Delta x^+ = A \Delta x + B \Delta u , \quad
\Biggl\{\begin{aligned} A &= I+T_s\frac{\partial f}{\partial  x}(\hat
  x,\hat u) \\
B &= T_s \frac{\partial f}{\partial u}(\hat x,\hat u)
\end{aligned}
\]
Although this system has a control input, the algorithm for computing the
MRPI set is applicable since we consider the closed loop system under
linear feedback  $u=Kx$, where $K$ satisfies the robust Lyapunov condition $V(x)-V((A+BK)x+v)\leq \gamma^2v^Tv$ 
with $V(x)=x^T P x\geq0$, i.e. $x^TPx - ((A+BK)x+v)^TP((A+BK)x+v)\geq x^T(Q+K^TRK)x -\gamma^2 v^Tv$
for a minimum  $\gamma^2$, see e.g.~\cite{Boyd:94}.
A representation of additive disturbances acting on the linearised
model due to linearisation errors, i.e. a system representation $x^+=Ax + Bu + v$ with an additive disturbance $v$, can be obtained using the
Mean Value Theorem for vector-valued functions (e.g.~\cite{Apostol:1974}).
%
%
\begin{thm}[Mean Value Theorem]\label{thm:mean:value:theorem}
Let $g : \mathcal X \rightarrow\mathbb R^m$ be continuously
differentiable, $\mathcal X\subset\mathbb R^n$ be open,
and $x \in\mathcal X$, $h \in\mathbb R^n$ be such that 
$x + th \in\mathcal X$ for all $t\in [0 ,1]$. Then
\begin{equation}
	g(x+h) = g(x) + \left(\int_0^1 \frac{\partial g}{\partial x}(x+th)dt\right)\cdot h.
\end{equation}
\end{thm}
%
%
Using the Mean Value Theorem
and the linearisation around $(\hat{x},\hat{u})$, and defining 
$\tilde{x} = x - \hat{x}$, $\tilde{u} = u - \hat{u}$ we obtain the successor state:
%
\begin{align*}
x^+ &= \tilde{f}(\hat{x} + \tilde{x},\hat{u} + \tilde{u}) 
\\
&= \tilde f(\hat x, \hat u) 
+ \int_0^1\frac{\partial\tilde f}{\partial x}(\hat x + t\tilde x,
\hat u + t\tilde u) dt \cdot \tilde x  \\
&\quad + \int_0^1\frac{\partial\tilde f}{\partial u}(\hat x + t\tilde x,\hat u+
t\tilde{u}) dt\cdot \tilde{u} \\
\Leftrightarrow 
\tilde x^+ &= A\tilde{x} + B \tilde{u} +\Bigl(
\int_0^1\frac{\partial\tilde f}{\partial x}(\hat x + t\tilde x,\hat u
             + t\tilde{u})dt - A \Bigr)\tilde x 
\\ 
&\quad +\Bigl(\int_0^1\frac{\partial\tilde f}{\partial u}(\hat x +
  t\tilde x,\hat u + t\tilde{u})dt - B\Bigr)\tilde{u}
\end{align*}
%
This implies the dynamics
%
\[
\tilde x^+ = A\tilde x+B\tilde{u} + H^x\tilde{x} + H^u \tilde{u}, 
\]
%
where $H^x$ and $H^u$ can be determined by integration. 
Suppose that for all $x\in\mathcal X$ and $u\in\mathcal U$, where
$\mathcal X$ and $\mathcal U$ are compact sets, $f(x,u)$ is
continuously differentiable. Then
we can bound the values of $H^x \tilde{x}+ H^u \tilde{u}$ by a finite convex combination of extremal values, i.e. 
$H^x \tilde{x}+H^u \tilde{u}\in\conv_k\{H^x_k \tilde{x} + H^u_k \tilde{u}\}$.
Clearly, we can then introduce the element\-wise disturbance bound 
%
\begin{multline}\label{eq:definition:element:wise:constraints:on:nonlinearities}
\mathcal V(\tilde{x},\tilde{u})=\biggl\{v:\min_k\{
H^x_{k,i}\tilde{x}+H^u_{k,i}\tilde{u}\}\leq v_i\,\wedge 
\\ 
v_i \leq \max_k\{H^x_{k,i}\tilde{x}+H^u_{k,i}\tilde{u}\}, \, i =1,\dots,n\biggr\}.
\end{multline}
%
With this set we can
guarantee that $\tilde x^+ = A\tilde x + B\tilde u + v$ accounts for all nonlinearities within $\mathcal X
\times\mathcal U$ if we constrain $v\in\mathcal V(\tilde x,\tilde u)$. For general nonlinear systems
finding the extremal values of $(H^x,H^u)$ cannot be done easily.
To obtain values for $(H^x_k,H^u_k)$ we sample $\mathcal X\times\mathcal U$ and evaluate the integral 
expressions defining $(H^x_k,H^u_k)$ pointwise.
This leads to an inner approximation of the linearisation error set, however we can make it as tight as necessary
by increasing the number of samples.
For this we use the numerical values for the example of the 
levitating ball: $Ts=30ms, C=1, m=100g, \hat x_1 = 50mm$ and $\mathcal X=\{x:
\abs{x_1- \hat x_1}\leq 1mm\wedge \abs{x_2}\leq 105\frac{mm}{s}\}$, $\mathcal U=\{u:\abs{ u-\hat u}\leq10mA\}$.
Using a total of 25 samples for the computation of $(H^x,H^u)$ we obtain the invariant set shown in figure~\ref{fig:MRPI:set:levitating:ball},
which is less conservative than using fixed bounds on the
nonlinearities as we will see in Section~\ref{sec:second:example}.
The algorithm for computing the MRPI set terminates after 3 iterations.
%
%
\begin{figure}
\centering
\begin{lpic}{invariantSetStateDependant(.65,)}
{\tiny
\lbl[r]{9,94; $0.1$}
\lbl[r]{9,86; $0.08$}
\lbl[r]{9,78; $0.06$}
\lbl[r]{9,70; $0.04$}
\lbl[r]{9,63; $0.02$}
\lbl[r]{9,55; $0$}
\lbl[r]{9,47; $-0.02$}
\lbl[r]{9,39; $-0.04$}
\lbl[r]{9,32; $-0.06$}
\lbl[r]{9,24; $-0.08$}
\lbl[r]{9,16; $-0.1$}
\lbl[t]{11,9; $-6$}
\lbl[t]{30,9; $-4$}
\lbl[t]{48,9; $-2$}
\lbl[t]{67.5,9; $0$}
\lbl[t]{86,9; $2$}
\lbl[t]{104,9; $4$}
\lbl[t]{123,9; $6$}
\lbl{120,3; $\times10^{-3}$}
}
{\small
\lbl{68,3; $\tilde x_1$}
\lbl{0,55,90; $\tilde x_2$}
}
\end{lpic}
\caption{The maximal robust positively invariant set for the levitating ball system.}
\label{fig:MRPI:set:levitating:ball}
\end{figure}
%
%
%
%
%
\section{Maximal Robust Positively Invariant Sets for Parametrised Disturbance}\label{sec:MRPI:parametrised}
%
%
In this section we describe the computation of the MRPI set for~\eqref{eq:system:equation} with a disturbance set 
parametrised by a scalar $\theta$:
%
\[
\mathcal V(\theta) = \{v: Gv\leq(1+\theta){\bf{1}}\} 
= (1+\theta)\mathcal V(0), \ \ \theta>-1.
\]
%
For input constrained systems under a given feedback law, we combine the uniform scaling of 
the disturbance set $\mathcal V(\theta)$ with non-uniform scaling of the input constraint set
%
\[
\mathcal U({\bf{\alpha}}) = \{u: Fu\leq\left(I+\text{diag}(\alpha_1,\dots,\alpha_p)\right){\bf{1}}\}.
\]
%
The necessity of uniform scaling of the disturbance constraints is due to the method we use to avoid
solving multi-parametric linear programs in every step of the proposed iteration.
A representation of the MRPI set of a system parametrised with respect to a scaling parameter allows us
to study the system's sensitivity to \emph{stronger/weaker} disturbances; similarly analysing the sensitivity to scaling of input 
constraints can be useful to choose particular actuators.
%
\begin{rem}
The set $\mathcal V(\theta)$ is non-empty and contains the origin for all $\theta>-1$ and hence the 
maximum
%
\[
0<\begin{array}[t]{rl}
\displaystyle\max_v & c^T v\\
\text{s.t.}& Gv \leq (1+\theta){\bf{1}}
\end{array}
\]
%
is positive for any non-zero $c$.
\end{rem}
%
In the following we describe an algorithm to compute the MRPI set $\mathcal Z^\infty$ contained in 
$\mathcal Z = \{(x,\theta):\mathcal F_i x+\mathcal G_i\theta \leq 1,\forall\, i\in\mathcal I\}$.
As in the state dependent case we iteratively introduce constraints \emph{separating} points
for which the successor state can lie outside the previous set, i.e. starting from $Z_0 = \mathcal Z$
we determine the first iterate by enforcing all individual constraints onto all possible successor states:
$Z_1=Z_0\cap D_0$ where $D_0$ is defined by
%
\begin{equation}\small
\begin{split}
	D_0 =& \{\mathcal F_i(\Psi x+v) + \mathcal G_i\theta \leq 1 \ \forall\, v\in\mathcal V(\theta), \ i\in\mathcal I_0\}\\
	=&\Biggl\{\mathcal F_i\Psi x + \begin{array}[t]{rl}\displaystyle\max_v& F_i v \\ \text{s.t.}& Gv \leq 
	(1+\theta){\bf{1}} \end{array}
	 + \mathcal G_i \theta \leq 1 \ \forall\, i\in\mathcal I_0\Biggr\}\\
	=&\Biggl\{\mathcal F_i\Psi x + (1+\theta)\underbrace{\begin{array}[t]{rl}\displaystyle\max_v& F_i v \\ 
	\text{s.t.}& Gv \leq {\bf{1}}\end{array}}_{v_{0,i}^\ast}
	 + \mathcal G_i \theta \leq 1 \ \forall\, i\in\mathcal I_0\Biggr\}\\
	=&\{\mathcal F_i\Psi x + (\mathcal G_i + v_{0,i}^\ast)\theta \leq 1 - v_{0,1}^\ast \ \forall \, i\in\mathcal I_0
	\}
\end{split}\end{equation}
%
Using the same principle we define $Z_{k+1}=Z_k\cap D_k$ with $D_k$ given by
%
\begin{equation}\label{eq:parametrised:Dk}\small
	D_k = \biggl\{\mathcal F_i\Psi^{k+1}x+ \biggl(\mathcal G_i +\sum_{0\leq l\leq k} v_{l,i}^\ast\biggr)\theta \leq 1 
	- \sum_{0\leq l\leq k} v_{l,i}^\ast\, i\in\mathcal I_0\biggr\}
\end{equation}
%
where we use 
%
\begin{equation}\label{eq:definition:v:ast}
v_{l,i}^\ast = \begin{array}[t]{rl}\displaystyle\max_v & \mathcal F_i \Psi^{l} v \\ \text{s.t.}& Gv\leq{\bf{1}}
\end{array}
\end{equation}
%
Notice that~\eqref{eq:definition:v:ast} can be represented in various ways:
%
\[
\small
\begin{array}{rlcrlcrl}
\max_v&\mathcal F_i \Psi^l v &=& \max_{\tilde v}&\mathcal F_i \tilde  v &=& \max_{\tilde v}&\mathcal F_i \tilde v \\ 
\text{s.t.}&v\in\bar{\mathcal{V}} & & \text{s.t.}&\Psi^{-l} \tilde v\in\bar{\mathcal{V}} & &
\text{s.t.}& \tilde v\in\Psi^l\bar{\mathcal{V}}
\end{array}
\]
%
where $\bar{\mathcal{V}}=\mathcal V(0)$ for notational convenience. For any fixed $\hat\theta>-1$
we have the closed form description
%
\begin{equation}\label{eq:closed:form:parametrised:iterates}
\begin{split}
	Z_{k+1}\vert_{\hat\theta} &= Z_{k}\vert_{\hat\theta}\cap\left(
	\Psi^{-1}Z_k\vert_{\hat\theta}\ominus\Psi^{k-1}\mathcal V(\hat\theta)\right)\\
	&=\bigcap_{0\leq l\leq k}\left(\Psi^{-l}\mathcal Z\vert_{\hat\theta}\underset{0\leq i\leq l-1}{\bigominus} 
	\Psi^i \mathcal V(\hat\theta)
	\right).
\end{split}
\end{equation}
%
The iteration terminates when $Z_k\subseteq Z_{k+1}$. As in section~\ref{sec:state:dep:MRPI}
we will require $\mathcal Z\vert_{\hat\theta}$ to be contained in an observable band: $\mathcal Z\vert_{\hat\theta}
=\{x:\mathcal Fx \leq {\bf{1}}-\mathcal G\hat\theta\}\subseteq \mathcal B=\{x:\Gamma x\leq {\bf{1}}\wedge -\Gamma 
x\leq {\bf{1}}\}$ for all $\hat\theta>-1$. We have the following result:
%
\begin{thm}
Let the system constraints be contained in a band $\mathcal Z\vert_{\hat\theta}\subseteq\mathcal 
B=\{x:\Gamma x\leq {\bf{1}}\wedge -\Gamma x\leq {\bf{1}}\}$ for any fixed $\hat\theta$, let the
pair $(\Psi,\Gamma)$ be observable and let $\mathcal V(0)$ be bounded, then $Z_N\subseteq Z_{N+1}$
for a finite $N$. Hence the MRPI set $\mathcal Z^\infty = Z_N$ is a finite polyhedron.
\end{thm}
%
\begin{proof}
The proof is similar to that of Lemma~\ref{thm:finite:MRPI:set:state:dependable}.
First we argue that for each fixed $\hat\theta$ the set $Z_k\vert_{\hat\theta}$ becomes compact using the same
observability argument as in Lemma~\ref{thm:finite:MRPI:set:state:dependable}.
We then use the representation~\eqref{eq:closed:form:parametrised:iterates} to argue that $D_k$ as
in~\eqref{eq:parametrised:Dk} grows exponentially and therefore contains any compact set after a finite number
of iterations. Fixing $\theta=\hat\theta$ does not affect the result since
our argument is constructed for a fixed matrix $\Gamma$ with rows scaled by 
$\frac{1}{1-\mathcal G_i\hat\theta}$; clearly we can re-scale the rows to accommodate any other choice of $\hat\theta>-1$.
\end{proof}

The iterative process described above allows uniform scaling of the disturbance set. As mentioned earlier, the 
algorithm can be extended to the case of non-uniformly scaled input constraints to accommodate more degrees of freedom for 
system analysis simply by using 
%
\[
\mathcal Z = \{(x,\theta,\alpha): \mathcal F x + \mathcal G\theta + 
\mathcal H \alpha \leq {\bf{1}}\}.
\]
%
This does not affect the algorithm since at each step of the iteration 
the elements of $\mathcal H$ remain unchanged. 
%
%
%
%
%
%
\section{Examples}\label{sec:second:example}
%
%
In this section we compute MRPI sets for the levitating ball system presented in 
section~\ref{sec:first:example} and for a purely numerical model. These examples illustrate the effectiveness 
of the proposed approach for state dependent disturbances and the use of parametrised MRPI sets as a system analysis tool. 

First we present the parametrised MRPI
set for the levitating ball. To obtain results comparable to those in Section~\ref{sec:first:example} we derive fixed bounds
on the effect of nonlinearities on each state by finding a fixed set containing the set $\mathcal V(x,u)$ 
in~\eqref{eq:definition:element:wise:constraints:on:nonlinearities} for all $(x,u)\in\mathcal X\times\mathcal U$. 
This yields a constant set $\mathcal V$ which is non-symmetric around the origin due to nonlinearity. We also introduce
a scaling parameter $\alpha$ such that $\mathcal U(0)=\mathcal U$ for $\alpha=0$. For this setup we obtain the MRPI
set $\mathcal Z^\infty=\{(x,\theta,\alpha): \Lambda_i^x x + \Lambda_i^\theta \theta + \Lambda_i^\alpha \alpha\leq
\lambda_i\,\forall i\leq m_\infty\}$. Notice that since $\alpha$ was introduced as a scaling parameter
for $\mathcal U(\alpha)=\{u: Fu\leq (1+\alpha){\bf{1}}\}$ all entries in $\mathcal H$ are non positive and
remain unchanged throughout the computation of $\mathcal Z^\infty$ so that $\Lambda^\alpha$ has only non
positive entries, i.e. increasing $\alpha$ will enlarge the parametrised MRPI set $\mathcal Z^\infty\vert_{\alpha_1}
\subseteq\mathcal Z^\infty\vert_{\alpha_2}$ for $\alpha_1\leq\alpha_2$.

First we compare the MRPI set $\mathcal X^\infty$ obtained with state dependent disturbance 
constraints with the parametrised set $\mathcal Z^\infty$. To do this we compute the scaling parameter 
$\alpha_{\min}$ such that $\mathcal X^\infty\subseteq\mathcal Z^\infty\vert_{(\theta=0,\alpha=\alpha_{\min})}$
by solving $m_\infty$ linear programs
%
%
\[
	\gamma_i^\ast = \begin{aligned}[t]
	\max_x & \ \Lambda_i^x x\\
	\text{s.t.}& \ x\in\mathcal X^\infty
	\end{aligned}
\]
%
%
The minimal value for $\alpha_{\min}$ is then given by the maximal $\alpha$ satisfying $\gamma_i^\ast + 
\Lambda_i^\theta\cdot 0 + \Lambda_i^\alpha \alpha \leq \lambda_i$. Solving this we obtain the 
minimal value $\alpha_{\min} = 1.7555$ and the MRPI sets shown in Figure~\ref{fig:minimal:scaling:comparison:MRPIs}.
The computation terminates after seven iterations and produces a polyhedron $\mathcal Z^\infty$ supported by $m_\infty=10$
planes.
Caution is advised when interpreting $\alpha>0$, since the system is nonlinear.
extrapolations reveal little insight, however $\alpha_{\min} > 0$ indicates that the current set of inputs $\mathcal U(0)=\mathcal U$ 
is not large enough to cope with the perturbation set given by upper bounds of the nonlinear effects.
%
%
\begin{figure}
\begin{lpic}{minimalAlphaForConstantDisturbance(.65,)}
{\tiny
\lbl[r]{7,87;$1.5$}
\lbl[r]{7,69;$1$}
\lbl[r]{7,51;$0.5$}
\lbl[r]{7,34;$0$}
\lbl[r]{7,16;$-0.5$}
\lbl[t]{15,2;$-0.06$}
\lbl[t]{27,2;$-0.05$}
\lbl[t]{39,2;$-0.04$}
\lbl[t]{51,2;$-0.03$}
\lbl[t]{63,2;$-0.02$}
\lbl[t]{75,2;$-0.01$}
\lbl[t]{89,2;$0$}
\lbl[t]{100,2;$0.01$}
\lbl[t]{113,2;$0.02$}
}
{\footnotesize
\lbl[r]{5,45,90;$\tilde x_2$}
\lbl[t]{121,3;$\tilde x_1$}
}
\lbl[l]{93,32;$\longleftarrow\mathcal X^\infty$}
\lbl[r]{68,43;$\mathcal Z^\infty\vert_{(0,\alpha_{\min})}\longrightarrow$}
\end{lpic}
\caption{The minimal scaling of the constant disturbance set such that the associated MRPI set contains
the MRPI set for piecewise affine disturbances is $\alpha_{\min}=1.7555$. Note that since the nonlinearities of the system
 dynamics are not symmetrical around the equilibrium the MRPI set with fixed constraints on the disturbance
 is non-symmetric.}
 \label{fig:minimal:scaling:comparison:MRPIs}
\end{figure}


In the second example, we compute the parametrised MRPI set for the system
%
$$
x^+ = \underbrace{\left(\begin{array}{cc}1 & 2 \\ 0 & 1\end{array}\right)}_Ax + 
\underbrace{\left(\begin{array}{cc} 1 & 1 \\ 1 & -1\end{array}\right)}_B u + 
\underbrace{\left(\begin{array}{cc} 1 & 2 \\ 2 & 1 \end{array}\right)}_D w.
$$
%
We use the constraint sets $\mathcal U(\alpha) = \{u: Fu\leq(I+\text{diag}(\alpha)){\bf{1}} \wedge
-Fu\leq (I+\text{diag}(\alpha)){\bf{1}}\}$ and $\mathcal V(\theta) = D\mathcal W(\theta)$ where $\mathcal W(\theta) = \{w:
\abs{w_1}\leq 0.1(1+\theta)\wedge \abs{w_2}\leq 0.15(1+\theta)\}$. The constraint matrix $F$ is given by
%
$$
	F = \left(\begin{array}{cc} 2 & 2 \\ 4 & 0 \\ 2 & -2 \end{array}\right).
$$
%
The constraints of this example were chosen for  illustrative purposes.
We now analyse the effects of changing individual constraints on the control input (by changing $\alpha$) as well as the
effect of scaling the perturbations $w$ (through $\theta$). To initialise the iteration we use
$\mathcal Z(x,\theta,\alpha) = \left\{(x,\theta,\alpha): Kx\in\mathcal U(\alpha)\wedge K_w x\in\mathcal W
(\theta)\wedge \alpha,\theta>-{\bf{1}}\right\}$, for $K$ and $K_w$ satisfying the robust Lyapunov condition
$x^TPx - ((A+BK)x+Dw)^TP((A+BK)x+Dw)\geq x^T(Q+K^TRK)x -\gamma^2 w^Tw$, i.e. $K_w = (\gamma^2-D^TPD)D^TP(A+BK)$.
We use $Q = \text{diag}(2,1)$ and $R = I$. Using these numerical values we determine the MRPI set 
$\mathcal Z^\infty$. After 22 iterations the algorithm terminates producing a total of 61 facets. 
By examining the conditions on $\mathcal H$ and the scaling parameters $\alpha$, it is possible to show that 
$\Lambda^\alpha$ can have at 
most one nonzero entry per row. We can therefore 
compute which $\alpha_i$ will change the MRPI set the most. This can be done elementwise by calculating
%
$$
	\max_i \  \frac{-\Lambda_i^\alpha}{\norm{\Lambda_i^x}_2}.
$$
%
For this example the greatest sensitivity corresponds to $\alpha_3$, 
i.e. the third input constraint.
Similarly to the first example we might also want to know how much disturbance the closed loop system can tolerate 
such that a given set is contained in the MRPI set $\mathcal C\subseteq\mathcal Z^\infty\vert_{(\theta,0)}$.
Although non-positivity of $\mathcal H$ implies non-positivity of $\Lambda^\alpha$, a similar argument cannot be applied to $\Lambda^\theta$.
Let $(\Lambda^1,\lambda^1)$ and $(\Lambda^2,\lambda^2)$ denote all rows of $\Lambda^\theta$ such that $\Lambda^{1,\theta}>0$ and
$\Lambda^{2,\theta}\leq0$. This implies that for any fixed $\hat\alpha$ the set $\mathcal Z^\infty\vert_{\hat\alpha}$
is compact, and since $\mathcal Z^\infty\vert_{(\hat\theta,\hat\alpha)}$
is compact, there exist a maximal $\theta_{\max}$ and likewise a minimal $\theta_{\min}$
such that $\Lambda^{1,x}x+\Lambda^{1,\alpha}\hat\alpha+\Lambda^{1,\theta}\theta\leq\lambda^1$ can be satisfied. We can solve the optimisation programs
%
$$
	\gamma_i = \begin{array}[t]{rl}
	\displaystyle\max_x& \Lambda^{1,x}_ix\\
	\text{s.t.}& x\in\mathcal C
	\end{array}
	\qquad
	\delta_i = \begin{array}[t]{rl}
	\displaystyle\max_x& \Lambda^{2,x}_ix\\
	\text{s.t.}& x\in\mathcal C
	\end{array}
	.
$$
%
The extremal values for which $\mathcal C$ is contained in the MRPI set are given by
the smallest $\theta_{\max}$ satisfying $\gamma_i+\Lambda^{1,\alpha}_i\hat\alpha+
\Lambda^{1,\theta}_i\theta_{\max}\leq\lambda_i^1$ and the largest $\theta_{\min}$ 
satisfying $\delta_i+\Lambda^{2,\alpha}_i\hat\alpha+\Lambda^{2,\theta}_i\theta_{\min}\leq\lambda_i^2$
The numerical values for the example are given by $[\theta_{\min},\theta_{\max}]=[-0.9999,6.2582]$.
Two- and three-dimensional illustrations of the parametrised MRPI set are given in Figure~\ref{fig:two:dim:example}
and~\ref{fig:three:dim:example} respectively.
%
%
%
\begin{figure}
\centering
\begin{lpic}{twoDimensional(.6,)}
{\tiny
\lbl[r]{4,9;$-3$}
\lbl[r]{4,22;$-2$}
\lbl[r]{4,35;$-1$}
\lbl[r]{4,49;$0$}
\lbl[r]{4,62;$1$}
\lbl[r]{4,76;$2$}
\lbl[r]{4,89;$3$}
\lbl[t]{6,3;$-10$}
\lbl[t]{17,3;$-8$}
\lbl[t]{28,3;$-6$}
\lbl[t]{39,3;$-4$}
\lbl[t]{50,3;$-2$}
\lbl[t]{63,3;$0$}
\lbl[t]{74,3;$2$}
\lbl[t]{85,3;$4$}
\lbl[t]{96,3;$6$}
\lbl[t]{107,3;$8$}
\lbl[t]{118,3;$10$}
}
{\footnotesize
\lbl[r]{0,52,90;$x_2$}
\lbl[t]{63,0;$x_1$}
}
\end{lpic}
\caption{The extremal MRPI sets $\mathcal Z^\infty\vert_{(\theta_{min},0)}$ and $\mathcal Z^\infty
\vert_{(\theta_{\max},0)}$ containing the unit box.}
\label{fig:two:dim:example}
\end{figure}
\begin{figure}
\centering
\begin{lpic}{threeDimensional(.6,)}
{\tiny
\lbl[r]{4,69;$6$}
\lbl[r]{4,61;$5$}
\lbl[r]{4,53;$4$}
\lbl[r]{4,45;$3$}
\lbl[r]{4,37;$2$}
\lbl[r]{4,29;$1$}
\lbl[r]{4,21;$0$}
\lbl[r]{4,13;$-1$}
\lbl{9,5;$6$}
\lbl{24,5;$4$}
\lbl{40,4;$2$}
\lbl{55,4;$0$}
\lbl{69,3;$-2$}
\lbl{83,3;$-4$}
\lbl{99,3;$-6$}
\lbl{116,8;$-10$}
\lbl{120,16;$0$}
\lbl{126,23;$10$}
}
{\footnotesize
\lbl{125,15;$x_1$}
\lbl{50,2;$x_2$}
\lbl[r]{-1,40,90;$\theta$}
}
\end{lpic}
\caption{The parametrised MRPI set for $\alpha=0$. Containing the unit box in the 
interior between $\theta_{\min}$ and $\theta_{\max}$.}
\label{fig:three:dim:example}
\end{figure}
%
%
%
\section{Conclusions}\label{sec:conclusions}
%
In this paper we discussed extensions to existing computational methods to determine MRPI sets for linear systems subject 
to additive disturbance for two descriptions of the sets bounding unknown additive disturbances. The case of state dependent 
disturbances was considered, and applied to the problem of determining approximations to MRPI sets for linearised nonlinear 
systems. This was illustrated using the example of a magnetic levitation system. Secondly the computation of MRPI sets for 
systems with disturbance and control inputs bounded by scaled sets was considered. This enables system analyses based on 
uniform scaling applied to disturbance sets and non-uniform scaling of input constraints. A comparison of the two approaches 
shows the effectiveness of the method based on state dependent disturbance sets. 

\bibliographystyle{IEEEtran} \bibliography{IEEEabrv,MyLib}
%
\end{document}