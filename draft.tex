\documentclass[letterpaper, 10pt, conference]{ieeeconf/ieeeconf} % Change font size to 9pt

\IEEEoverridecommandlockouts
\overrideIEEEmargins

\usepackage{graphics}
\usepackage{epsfig}
% \usepackage{mathptmx}
% \usepackage{times}
\usepackage{amsmath}
\usepackage{amssymb}
\usepackage{cite}
\usepackage{lpic}

\newtheorem{thm}{Lemma}[section]
\newtheorem{rem}[thm]{Remark}
\newtheorem{defn}[thm]{Definition}
\newtheorem{assn}{Assumption}
\newtheorem{algo}[thm]{Algorithm}

\providecommand{\norm}[1]{\left\|#1\right\|}
\providecommand{\abs}[1]{\left|#1\right|}
\providecommand{\conv}{\text{conv}}

\usepackage{xcolor}
\def\edit{\textcolor{blue}}
\allowdisplaybreaks[4]


\DeclareFontFamily{U}{mathx}{\hyphenchar\font45}
\DeclareFontShape{U}{mathx}{m}{n}{
      <5> <6> <7> <8> <9> <10> gen * mathx
      <10.95> mathx10 <12> <14.4> <17.28> <20.74> <24.88> mathx12
      }{}
\DeclareSymbolFont{mathx}{U}{mathx}{m}{n}
\DeclareFontSubstitution{U}{mathx}{m}{n}
\DeclareMathSymbol{\temp}{\mathbin}{mathx}{'341}
\newcommand{\bigominus}{\raisebox{10pt}{$\temp$}}

\graphicspath{{./Pix/}}

\begin{document}
  \title{Advances in Robust Positive Invariant Sets Computation}

\author{Rainer M. Schaich\textsuperscript{\dag} %
         and Mark Cannon\textsuperscript{\dag,\ddag}%
\thanks{\textsuperscript{\dag} Department of Engineering Science, University of Oxford, OX1 3PJ.}%
\thanks{\textsuperscript{\ddag} Corresponding author, 
        \texttt{mark.cannon@eng.ox.ac.uk}.}
}


\maketitle

\begin{abstract} 

\end{abstract}

\begin{keywords}
\vskip-\baselineskip
\end{keywords}

\section{Introduction}
\begin{equation}\label{eq:system:equation}
	x^+ = \Psi x + v
\end{equation}
\begin{equation}\label{eq:definition:mrpi:set:state:dependent}
	\mathcal X^\infty = \{x:\Psi x + v\in\mathcal X^\infty\; \forall v\in\mathcal V(x)\}
\end{equation}
\begin{equation}\label{eq:PWA:distrubance:set}
	\mathcal V(x) = \left\{v: Gv\leq\max\{{\bf{1}},H^x x\}\right\}
\end{equation}
\section{Parametrically convex set operations}
In this section we discuss sets that depend on a state-like parameter (so called point-to-set maps, 
see~\cite{Hogan:1973}), and extend the existing set algebra~\cite{blanchini:2007} to 
accommodate such sets. We present the general case first, then derive the computation of
the case with piece-wise affine sets.


%First we introduce
    \begin{defn}[Parametric Convexity]\label{def:parametric:convexity}
      Let $X\subseteq\mathbb R^n$ and $Y\subseteq\mathbb R^m$ and $T:X\rightarrow Y$, $X\ni s\mapsto T(s)\subset Y$ be a 
      continuous point-to-set map. The map $T$ is called \emph{parametrically convex} if it satisfies
      \begin{equation}\label{eq:pconvexdef}
        T(\lambda s_1 + (1-\lambda)s_2)\subseteq\lambda T(s_1) \oplus (1-\lambda) T(s_2)
      \end{equation}
      for all $s_1,s_2\in X$ and $\lambda\in[0,1]$.
    \end{defn}

In (\ref{eq:pconvexdef}), $\oplus$ denotes the Minkowski set addition
      \begin{equation}
        \mathcal A\oplus\mathcal B = \{c : c = a + b\; \forall\,a\in\mathcal A,\, b\in\mathcal B\}.
      \end{equation}
%      see~\cite{blanchini:2007}.
%    Note also that Definition~\ref{def:parametric:convexity} does not require $X$ and $Y$ to be of equal dimensions.
    In the following we will need to be able to compute differences of sets
    analogous to the Pontryagin difference in~\cite{Kolmanovsky:1998} but using parametrised 
    sets, this is defined as follows:

    \begin{defn}[Parametric Pontryagin Difference]\label{def:parametric:pontryagin:difference}
Let $S\subseteq X$ and let $T:X\to X$ be a continuous point-to-set map,
%      Let $T$ be a continuous point-to-set map from $X$ to $X$, and let $S\subseteq X$, 
then the \emph{parametric Pontryagin difference} $S\ominus T(S)$ is 
      \begin{equation}
        S\ominus T(S) = \left\{x\in X: \{x\} \oplus T(x)\subseteq S\right\},
      \end{equation}
      where $T(S)$ denotes the image of $S$ under the map $T$.
    \end{defn}

    For parametric Pontryagin differences of convex sets and parametrically convex maps we have the following result.

    \begin{thm}\label{thm:convexity:of:pontryagin:difference}
      Let $S\subseteq X$ be a convex set and let $T:X\rightarrow X$ be a parametrically convex point-to-set
      map, then $S\ominus T(S)$ is convex. 
    \end{thm}
    \begin{proof}
      Define $ Z =  S\ominus T( S)$ and let $z_1,z_2\in Z$, then
      by definition of the parametric Pontryagin difference, we have
      \begin{equation}
        \{z_i\} \oplus T(z_i) \subseteq S,\; i=1,2.
      \end{equation}
      To see that $ Z$ is convex we show that line segments between
      all possible $z_1$ and $z_2$ are subsets of $ Z$, i.e.~for all $\lambda \in [0,1]$,
%      \begin{equation}\label{eq:proof:of:parametric:convexity}
      \[\begin{aligned}
        \{ \lambda z_1 + (1-&\lambda)z_2
        \}\oplus T\left( \lambda z_1 + (1-\lambda)z_2\right)\\
        \subseteq&\left\{ \lambda z_1 + (1-\lambda)z_2
        \right\}\oplus \lambda T(z_1) \oplus (1-\lambda)
         T(z_2)\\
        \subseteq &\lambda\underbrace{(\{z_1\}\oplus T(z_1))}_{\subseteq S}\oplus
        (1-\lambda)\underbrace{(\{z_2\}\oplus T(z_2))}_{\subseteq S}\\
        \subseteq& Z
        \end{aligned}\]
%      \end{equation}
      where the last inclusion follows from the convexity of $\mathcal S$.
    \end{proof}

    In the given setup we consider a point-to-set map defined by
    \begin{equation}\label{eq:definition:disturbance:set}
      \mathcal V(x) = \{v:\,G v \leq H(x)\}
    \end{equation}
    where $H$ is elementwise convex in $x$, so that 
    $H_i(\lambda x_1+(1-\lambda)x_2)\leq \lambda H_i(x_1)+(1-\lambda)H_i(x_2)$ for 
    $\lambda\in[0,1]$.  For such sets we have the following result.
    \begin{thm}\label{thm:convex:parametric:set}
      $\mathcal V (x)$ defined in~\eqref{eq:definition:disturbance:set} is parametrically convex.
    \end{thm}
    \begin{proof}
      To show that $\mathcal V(\lambda x_1 + (1-\lambda)x_2)\subseteq 
      \lambda\mathcal V(x_1) \oplus(1-\lambda)\mathcal V(x_2)$ for all $\lambda \in [0,1]$ we note that
        \begin{align*}
        \mathcal V&(\lambda x_1 + (1-\lambda)x_2)\\
        =& \{v:\; G v \leq H(\lambda x_1 + (1-\lambda)x_2)\}\\
        \subseteq& \{v:\;Gv\leq\lambda H(x_1)+(1-\lambda) H(x_2)\}\\
        =&\{v:\;Gv\leq\lambda H(x_1)\}\oplus\{v
        :\;Gv\leq(1-\lambda)H(x_2)\}\\
        =&\lambda\mathcal V(x_1)\oplus(1-\lambda)\mathcal V(x_2).
        \end{align*}
\vskip-1.5\baselineskip
    \end{proof} 
\def\genmat{\Xi} \def\genvec{\xi}Lemmas~\ref{thm:convexity:of:pontryagin:difference} 
and~\ref{thm:convex:parametric:set} imply that the difference of a convex set and 
$\mathcal V(x)$ is a convex set.
Furthermore, we will see that if $\mathcal X$ is polyhedral and $\mathcal 
V(x)$ is as defined in~\eqref{eq:PWA:distrubance:set}, then $\mathcal X\ominus\mathcal V(\mathcal X)$ is 
again a polytopic set. This will become more clear in the next section when we describe 
the computation of the mRPI set~\eqref{eq:definition:mrpi:set:state:dependent}.

\section{Maximal Robust Positive Invariant Sets for State Dependent Disturbance}
In this section we describe an iterative algorithm to compute the mRPI 
set~\eqref{eq:definition:mrpi:set:state:dependent} for a linear system~\eqref{eq:system:equation}.
For this notice that the set $\mathcal V(x)$ can be represented as the convex hull of its vertices 
$\mathcal V(x) = \conv\{v_i(x)\}$. Since ${\mathcal{V}}(x)$
has a piece-wise affine dependence on $x$ the vertices $v_i(x)$ are also piece-wise affine in $x$.
The set $\mathcal X^\infty$ as defined in~\eqref{eq:definition:mrpi:set:state:dependent} 
requires to satisfy $\Psi x + v\in\mathcal X^\infty$ for all
$x\in\mathcal X^\infty$ and $v\in\mathcal V(x)$, to compute the mRPI set we start from a given set
$\mathcal X_0 = \{x:\;\genmat_{0,i}x\leq \genvec_{0,i}\,\forall i\in\mathcal I_0\}$, e.g.
the system constraints, and 
\emph{cut off} all points that do not satisfy the invariance condition in one step, two steps, 
etc. until only points satisfying the conditions are left. I.e. we iteratively introduce 
constraints that cut off all points for which the successor state can lie outside 
$\mathcal X_0$. So for the first iteration we need to enforce the constraint:
\begin{equation}\begin{split}
	&\genmat_{0,i}(\Psi x + v)\overset{!}{\leq}\genvec_{0,i}\;\forall v\in\conv\{v_i(x)\}\\
	&\genmat_{0,i}\Psi x + \max_{v\in\mathcal V(x)} \genmat_{0,i} v \leq \genvec_{0,i}\\
	&\genmat_{0,i}\Psi x + \underbrace{\max_{j} \genmat_{0,i} v_j(x)}_{v_{0,i}^\ast(x)} \leq \genvec_{0,i}.
\end{split}\end{equation}
to each inequality.
Notice that $v_{0,i}^\ast(x)$ is not necessarily given by one unique maximiser, 
but is the solution of a multiparametric linear program and hence will be given 
by a vertex $v_i(x)$ of $\mathcal V(x)$ for each $x$ on that facet.
Since each vertex is a piece-wise affine function of $x$ the maximum $v_{0,i}^\ast(x)$ will also
be piece-wise affine
and therefore the set $\mathcal X_1=\mathcal X_0 \cap \{x:\genmat_{0,i}\Psi x + v_{0,i}^\ast(x) \leq 
\genvec_{0,i}\forall i\in\mathcal I_0\}$
has the representation $\mathcal X_1 = \{x:\genmat_{1,i}x\leq\genvec_{1,i}\,\forall i\in\mathcal I_1\}$.
The next iterate is defined by
\begin{multline}
	\mathcal X_2 = \mathcal X_1 \cap \\ \{x:\genmat_{0,i}\Psi(\Psi x + v) + v_{0,i}^\ast(x)\leq\genvec_{0,i}\,
	\forall i\in\mathcal I_0,v\in\mathcal V(x)\}\\
	= \mathcal X_1 \cap \{x: \genmat_{0,i}\Psi^2 x + v_{1,i}^\ast(x) + v_{0,i}^\ast(x)\leq\genvec_{0,i}\,\forall 
	i\in\mathcal I_0\}.
\end{multline}
And we analogously define
\begin{equation}
	\mathcal X_{k+1} = \mathcal X_k\cap \{x:\genmat_{0,i}\Psi^k x + \sum_{l=0}^{k-1}v_{l,i}^\ast(x)
	\leq\genvec_{0,i}\,\forall i\in\mathcal I_0\},
\end{equation}
where we have 
\begin{multline}
	v_{l,i}^\ast(x)=\begin{split}\max& \genmat_{0,i}\Psi^{l-1}v_j(x)\\ j&\end{split}
	 = \begin{split}&\max \genmat_{0,i}\Psi^{l-1}v \\ &v\in\mathcal V(x)\end{split}\\
	= \begin{split}&\max \genmat_{0,i}\tilde v\\ &\tilde v\in \Psi^{l-1}\mathcal V(x) \end{split}
\end{multline}
In a closed form the iterates can be expressed as
\begin{equation}\label{eq:set:iteration:state:dependent:constraints}
\begin{split}
	\mathcal X_{k+1} =& \mathcal X_k\cap\left(\Psi^{-1}\mathcal X_k \ominus \Psi^{k-1}\mathcal V(\mathcal X_k)\right)
	=\mathcal X_k\cap D_k \\
	=& \bigcap_{l\leq k+1}\left( \Psi^{-l} \mathcal X_0 \underset{i\leq l-1}{\bigominus} \Psi^i \mathcal V(\mathcal X_{l-1})\right).
\end{split}\end{equation}
We will use~\eqref{eq:set:iteration:state:dependent:constraints} to prove the finite determinability of
$\mathcal X^\infty$, i.e. that there exists a finite number $N$ such that $x\in\mathcal X_N$ implies
$\Psi x + v \in\mathcal X_N$ for all $v\in\mathcal V(x)$ and the set therefore is robustly positive invariant.
\begin{thm}
Let the system constraints be contained in a band $\mathcal X_0\subseteq B=\{x:\Gamma x\leq{\bf{1}}\wedge 
-\Gamma x\leq{\bf{1}}\}$, let the pair $(\Psi,\Gamma)$ be observable and let $\mathcal V(\mathcal X_0)$ be bounded, 
then $\mathcal X_N\subseteq \mathcal X_{N+1}$ for a finite $N$. Hence the mRPI set $\mathcal X^\infty 
=\mathcal X_N$ is a finite polytope.
\end{thm}
\begin{proof}
The proof has two main steps: First we will prove that $\mathcal X_p$ is compact for $p\leq n$
where $n$ is the state dimension. The second step is to prove that in~\eqref{eq:set:iteration:state:dependent:constraints}
the set $D_k$ grows exponentially, i.e. that for any given compact set $\mathcal C$ there exists
a $\tilde N$ such that $\mathcal C\subseteq D_{\tilde N}$. The proof is concluded by setting 
$\mathcal C = \mathcal X_p$ and using $N = \tilde N$. For the first step, notice that the observability of
$(\Psi,\Gamma)$ is equivalent to the observability matrix $\Omega$ having full rank, i.e.
\[
\Omega = \left(\begin{array}{c}
\Gamma\\ \vdots \\ \Gamma\Psi^{n-1}
\end{array}\right)
\]
having a trivial kernel. But this then implies that the set $\mathcal P_{n+1} = \{x: 
\Omega x\leq{\bf{1}}\wedge-\Omega x\leq{\bf{1}}\} = \bigcap_{l\leq n+1} \Psi^{-l} B$ is bounded.
Notice that the containment $\mathcal X_k\subseteq \mathcal P_k$ holds for all $k$ and hence
the set $\mathcal X_{n+1}$ is also bounded. The case that $\tilde N<n$ occurs when $\Gamma$
has more than rank one. To see that $D_k$ grows exponentially we use that $\mathcal V(\mathcal X_0)$
is bounded, let $K_1$ denote the smallest ball that contains $\mathcal V(\mathcal X_0)$ and let
$\bar\lambda$ denote the magnitude of the largest eigenvalue of $\Psi$. Furthermore, let $K_1$
be the biggest ball that is contained in $\mathcal X_{0}$. $D_k$ has the representation
\begin{equation}
D_k = \underbrace{\Psi^{-k}\mathcal X_0}_{(\ast)} \ominus \underbrace{\left(\bigoplus_{i\leq k-1} 
\Psi^i\mathcal V(\mathcal X_k)\right)}_{(\ast\ast)}.
\end{equation}
Since $\Psi$ is asymptotically stable we know that $\bar\lambda<1$ and so we have 
$\bar\lambda^{-l}K_1\subseteq\Psi^{-l} K_1 \subseteq \Psi^{-l}\mathcal X_0 = (\ast)$
which implies exponential growth of $D_k$ as long as $(\ast\ast)$ is bounded.
To bound $(\ast\ast)$ we use $(\ast\ast)\subseteq \oplus_{i\leq k-1} \bar\lambda^i K_2 \subseteq
\frac{1}{1-\bar\lambda} K_2$. This concludes the proof: We proved that $D_k$ grows exponentially,
that it contains an exponentially growing ball. We proved that the subtrahend is bounded inside a ball of
finite radius, and hence we conclude that after a finite number of iterations $N$ intersecting with $D_{N+1}$
will no longer change $\mathcal X_N$.
\end{proof}
We have seen that we can compute a the maximal robust positive invariant set $\mathcal X^\infty$ for linear 
systems with state dependent constraints. In the next section we will illustrate the concept with an 
example.
\section{Example}
In this section we discuss the calculation of the mRPI set for a simplified model of the levitating
ball system depict in figure~\ref{fig:levitating:ball}. The system dynamics for the ball are given
by $m \ddot y = m g - c\frac{i^2}{y^2}$, where $m,g,c,i$ and $y$ denote the mass of the ball, the gravitational
constant, a constant factor, the current and the distance between the coil and the centre of the ball respectively.
For illustration purposes we neglect inductive dynamics and use the current $u=i$ as an input and the position
$y$ and its first derivative $\dot y$ as the states, i.e. $x = (y,\dot y)^T$. We find that an equilibrium
is present when $x_2=0$ and $u=\sqrt{\frac{gm}{c}}x_1$ for any positive position $x_1>0$.
Linearising the nonlinear differential equation $\dot x = f(x,u)$ around $\hat x, \hat u$ we obtain
\begin{equation}
	\Delta \dot x = \underbrace{\left(\begin{array}{cc}
	0 & 1 \\ \frac{2c\hat u^2}{m\hat x_1^3} & 0
	\end{array}\right)}_{\frac{\partial f}{\partial x}(\hat x,\hat u)}\Delta x + \underbrace{\left(\begin{array}{c}
	0 \\ - \frac{2c\hat u}{m\hat x_1^2}
	\end{array}\right)}_{\frac{\partial f}{\partial u}(\hat x,\hat u)}\Delta u.
\end{equation}
We derive discrete time system dynamics using the explicit Euler formula $x^+=x+T_s f(x,u) =:\tilde f(x,u)$,
with the sampling rate $T_s$. We hence define the system matrix $A = (I+T_s\frac{\partial f}{\partial x}(\hat x,\hat u))$
and the input matrix $B = T_s \frac{\partial f}{\partial u}(\hat x,\hat u)$. In order to obtain a 
representation of additive disturbances depending on the state and input like~\eqref{eq:definition:disturbance:set}
we use the mean-value theorem for vector valued functions:
\begin{thm}
Let $\mathcal X\subset\mathbb R^n$ be open, $f : \mathcal X \rightarrow\mathbb R^m$ continuously differentiable, 
and $x \in\mathcal X, h \in\mathbb R^n$ vectors such that the 
whole line segment $x + th$ remains in $\mathcal X$ for $0 \leq t \leq 1$. Then
\begin{equation}
	f(x+th) = f(x) + \left(\int_0^1 \frac{\partial f}{\partial x}(x+th)dt\right)\cdot h.
\end{equation}
\end{thm}
\begin{figure}
\centering
\begin{lpic}{levitatingBall}
\lbl[tr]{25,3; $m g$}
\lbl[br]{25,25; $c\frac{i^2}{x^2}$}
\lbl[bl]{49,17; $y$}
\lbl[bl]{56,55; $i$}
\end{lpic}
\caption{Levitating ball system.}
\label{fig:levitating:ball}
\end{figure}
\section{Maximal Robust Positive Invariant Sets for Parametrised Disturbance}
\section{Example}

\bibliographystyle{IEEEtran}
\bibliography{IEEEabrv,MyLib}

\end{document}